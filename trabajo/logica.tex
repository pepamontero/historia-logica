
\documentclass{article}
\usepackage{graphicx} % Required for inserting images
\usepackage{amsfonts}

\title{Lógica computacional?}
\author{Pepa Montero Jimena}
\date{November 2024}

% Diseño del párrafo
\setlength{\parindent}{0pt}
\setlength{\parskip}{.8em}

% Espaciados entre palabras en el justificado.
\sloppy

% Quotes
\usepackage{csquotes}



\begin{document}

\maketitle

\section{Introducción}

Motivación

\section{Nacimiento de la lógica}

Como hemos visto en clase, las matemáticas hasta este momento no tienen muchas pruebas formales (que sepamos).

[aquí faltan cosas y tal]

Es en el momento de la Historia de Grecia en la que empieza a aparecer un interés por la prueba formal en matemáticas. Este interés se intensifica con la aparición en el siglo VI A.C. de la democracia en la \textit{polis} griega de Atenas, sistema en el que todos los ciudadanos podían participar directamente en las decisiones relativas a la governanza de la ciudad.

Las reuniones de la asamblea, un órgano político donde se reunían los ciudadanos, eran uno de los eventos más importantes en Atenas, llegando a asistencias de hasta 6.000 personas. En estas asambleas se planteaba un objeto de debate, alrededor del cual distintos participantes podían exponer un discurso defendiendo su posición, ya fuera a favor o en contra. Después, todos los presentes votaban 'sí' ó 'no' a la cuestión planteada.

En este contexto, la habilidad de un ciudadano para ser un orador persuasivo empieza a cobrar una gran importancia. Consecuentemente, aquellos que podían enseñar esta habilidad a los ciudadanos tenían algo muy valioso que ofrecer. Muchos de los pensadores conocidos en esta época se convirtieron precisamente en profesores de esta ?, a los que se les empezó a llamar \textit{sofistas} de forma colectiva.

La idea es, por tanto, que las pruebas deductivas en el contexto matemático surgen inicialmente como un medio de persuasión en un contexto dinámico y colectivo. Un ejemplo en el que podemos ver esta intención reflejada claramente es la famosa prueba de la duplicación del cubo en el Menón de Platón \cite{novaes2020dialogical}; lejos de ser de una prueba formal, se trata de una conversación entre dos personajes. El maestro intenta convencer al otro de su punto de vista, en este caso, la validez de su método para construir el cuadrado con área doble.

\begin{displayquote}
SOKRATES. Entonces, ¿cuál es la extensión de este último? ¿No será cuatro veces mayor?\\
EL ESCLAVO. Forzosamente.\\
SOKRATES. ¿Y es que tal vez una cosa cuatro veces más grande que otra es el doble que ella?\\
EL ESCLAVO. ¡No, por Zeus!\\
SOKRATES. ¿Qué será, pues?\\
EL ESCLAVO. El cuádruplo.\\
SOKRATES. Por consiguiente, doblando la línea, no obtienes una superficie doble, sino cuádruple.\\
EL ESCLAVO. Es verdad. \cite{bergua1958dialogos}
\end{displayquote}

Sin embargo, debido al aumento de la importancia del medio escrito, ocurre una transformación desde las situaciones dialógicas concretas, que involucran a participantes reales, hacia un enfoque abstracto e impersonal, en el que se considera que todas las premisas necesarias para derivar una conclusión deben ser explícitas. Esta transformación ocurre como una reacción contra lo que se percibía como los modos meramente persuasivos cultivados por oradores, sofistas y políticos en el contexto del debate público. Podemos decir, entonces, que la deducción emerge como una respuesta diferente, supuestamente mejor, a necesidades y funciones similares relacionadas con el discurso racional y la persuasión en la vida pública. \cite{novaes2020dialogical}

Alrededor del 350 A.C., Aristóteles publica(?) la obra \textit{Primeros analíticos}, una de las s

Movida:
1. Explicar el Organon, las obras, y las principales contribuciones de Aristóteles

2. Explicar el impacto que tuvo aristóteles posteriormente (mucho, prácticamente hasta el siglo XIX no se utilizó nada más; ver la página de wikipedia del organon y tirar de ahi...)


\section{Siglos XIX, XX}

Orden cronológico de las cosas que están pasando aquí:

1. Lógica simbólica: desarrollo de la
lógica simbólica en los siglos XIX y XX, momento en el que ocurren varias
innovaciones, como por ejemplo la renovación de la lógica formal (Boole, De
Morgan), la aparición de la teoría de conjuntos (Cantor), la teoría de tipos
(Russel)

2. 1930: Kurt Gödel proved the Completeness Theorem, which states that if a formula is logically valid in first-order logic, it is provable in its formal system. This was part of his doctoral dissertation.

3. 1936: Alan Turing introduced the concept of the Turing machine in his paper "On Computable Numbers, with an Application to the Entscheidungsproblem (= Decidability problem)", laying the foundation for computability theory.

4. Apparition of Logic Programming (Planner and Prolog) 
1967: Planner, developed by Carl Hewitt, introduced concepts of logical inference and goal-directed reasoning., Prolog in 1972

5. 1971: Stephen Cook introduced the concept of NP in his landmark paper, "The Complexity of Theorem-Proving Procedures". This formalized the class of decision problems verifiable in polynomial time by a nondeterministic Turing machine.

Yo creo que este es un buen timeline y sería ir desarrollando estos conceptos y tal de forma que se vayan conectando entre sí y al concepto de la lógica



Aquí puede ser interesante el libro de Eloy?? No estoy segura en realida

Sería interesante hablar de :


\section{Theorem Provers??}
- Tecton
- Lean


% Bibliografía

\renewcommand{\refname}{Referencias}
\bibliographystyle{plain}
\bibliography{references.bib}

\end{document}
