
\documentclass{article}
\usepackage{graphicx} % Required for inserting images
\usepackage{amsfonts}
\usepackage{amsmath}

\title{Lógica computacional?}
\author{Pepa Montero Jimena}
\date{November 2024}

% Diseño del párrafo
\setlength{\parindent}{0pt}
\setlength{\parskip}{.8em}

% Espaciados entre palabras en el justificado.
\sloppy

% Quotes
\usepackage{csquotes}

% cSpell:disable

\begin{document}

\maketitle


\section{Introducción}

Motivación: ver el desarrollo de la lógica desde aristóteles hasta los primeros lenguajes de programación lógica, el camino que hemos tomado para llegar a computabilizar la lógica y también cuáles son los límites de esta formalización computacional

Ideas:
- en un principio hacer un breve sketch de historia de la lógica desde su "aparición" con aristoteles hasta el nacimiento de la logica simbolica . destacar que la logica de terminos siguio siendo la única realmente relevante? fundamentalmente hasta el siglo XVII (constrastar) y que no hay muchas innovaciones
- después notar que nos vamos a centrar solo en autores y momentos concretos que nos interesan para llegar a Prolog
mi idea fundamental del trabajo es llegar quizás con el mismo ejemplo desde aristóteles hasta prolog (estaría guay repetir este ejemplo a lo largo de las explicaciones de distintas lógicas, no tiene que ser un ejemplo complicado) y ver que se puede hacer con ello en prolog
- para llegar a prolog tenemos que pasar por una "matematización" de la lógica y también del desarrollo de la lógica proposicional. pero también destacar las similiaridades de prolog con la logica de aristoteles ()
- pongo en particular prolog como ejemplo fundamental de lenguaje lógico pero estaría guay generalizar esto
- como cosas potencialmente extras: comentar una relación entre la no computabilidad de problemas; comentar la relación con asistentes de demostración quizás.

\section{Nacimiento de la lógica}

¿Cuál es el origen de la lógica como disciplina? ¿Existe un momento concreto de la Historia en el que las personas aprendieron a diferenciar entre razonamientos sólidos de aquellos que no lo son? Es difícil responder a esta pregunta con exactitud, pues es fácil pensar que las personas llevan haciendo razonamientos lógicos (por lo menos en contextos relativamente sencillos) desde el inicio de los tiempos.

Podemos considerar que la lógica nace a raíz del creciente interés por presentar nuestras reglas de inferencia naturales como un sistema coherente. Si pensamos en la lógica como un sistema de reglas de inferencia justificables que separa los razonamientos válidos de aquellos que no lo son, podemos considerar a Aristóteles como el padre de esta disciplina\cite{moravcsik2004logic}. 

Sin embargo, no podemos considerar este surgimiento como un hecho aislado. Es evidente que era necesaria una base conceptual sólida y un desarrollo cognitivo relacionado con el interés en las matemáticas y la formalización, que llevaba ya tiempo gestándose en otros lugares a lo largo del tiempo. Sin embargo, el estudio de estos precedentes históricos escapa los límites de este trabajo.

\subsection{Lógica aristotélica}

\begin{displayquote}
    \textit{Todos los hombres son mortales. Todos los griegos son hombres. Luego todos los griegos son mortales.}
\end{displayquote}

Es en el momento de la Historia de Grecia en la que empieza a aparecer un interés por la prueba formal en matemáticas. Este interés se intensifica con la aparición en el siglo VI A.C. de la democracia en la \textit{polis} griega de Atenas, sistema en el que todos los ciudadanos podían participar directamente en las decisiones relativas a la gobernanza de la ciudad.

Las reuniones de la asamblea, un órgano político donde se reunían los ciudadanos, eran uno de los eventos más importantes en Atenas, llegando a asistencias de hasta 6.000 personas. En estas asambleas se planteaba un objeto de debate, alrededor del cual distintos participantes podían exponer un discurso defendiendo su posición, ya fuera a favor o en contra.

En este contexto, la habilidad de un ciudadano para ser un orador persuasivo empieza a cobrar una gran importancia. Consecuentemente, aquellos que podían enseñar esta habilidad a los ciudadanos tenían algo muy valioso que ofrecer. Muchos de los pensadores conocidos en esta época se convirtieron precisamente en profesores de esta competencia, a los que se les empezó a llamar \textit{sofistas} de forma colectiva.

La idea es, por tanto, que las pruebas deductivas en el contexto matemático surgen inicialmente como un medio de persuasión en un contexto dinámico y colectivo. Un ejemplo en el que podemos ver esta intención reflejada claramente es la famosa prueba de la duplicación del cubo en el Menón de Platón\cite{novaes2020dialogical}; lejos de ser de una prueba formal, se trata de una conversación entre dos personajes. Uno de ellos intenta convencer al otro de su punto de vista, en este caso, de la validez de su método para construir el cuadrado con área doble.

\begin{displayquote}
SOKRATES. Entonces, ¿cuál es la extensión de este último? ¿No será cuatro veces mayor?\\
EL ESCLAVO. Forzosamente.\\
SOKRATES. ¿Y es que tal vez una cosa cuatro veces más grande que otra es el doble que ella?\\
EL ESCLAVO. ¡No, por Zeus!\\
SOKRATES. ¿Qué será, pues?\\
EL ESCLAVO. El cuádruplo.\\
SOKRATES. Por consiguiente, doblando la línea, no obtienes una superficie doble, sino cuádruple.\\
EL ESCLAVO. Es verdad.\cite{bergua1958dialogos}
\end{displayquote}

Sin embargo, debido al aumento de la importancia del medio escrito, ocurre una transformación desde las situaciones dialógicas concretas, que involucran a participantes reales, hacia un enfoque abstracto e impersonal, en el que se considera que todas las premisas necesarias para derivar una conclusión deben ser explícitas. Esta transformación ocurre como una reacción contra lo que se percibía como los modos meramente persuasivos cultivados por oradores, sofistas y políticos en el contexto del debate público. Podemos decir, entonces, que la deducción emerge como una respuesta diferente, supuestamente mejor, a necesidades y funciones similares relacionadas con el discurso racional y la persuasión en la vida pública\cite{novaes2020dialogical}.

Alrededor del del 350 A.C., Aristóteles publica las obras \textit{Analytica Priora} y \textit{Analytica Posteriora}, cuyo objeto de estudio es la lógica demostrativa\cite{smith1989prior}. La demostración en el contexto de lógica demostrativa se entiende por tanto como una oposición a la persuasión: según Aristóteles, una demostración es una argumentación extendida que parte de unas premisas consideradas ciertas, e incluye una cadena de razonamientos que muestran mediante pasos deductivamente evidentes que la conclusión es una consecuencia de estas premisas. De esta forma, el objetivo de la demostración es la producción de conocimiento, al contrario que la persuasión, cuyo objetivo es la producción de creencias u opiniones\cite{corcoran2009aristotle}.

El objetivo de Aristóteles era, por tanto, presentar una teoría general de verdad y consecuencia que aplicara a todas las demostraciones. Sin embargo, para el propósito de \textit{Analytica Priora}, se centra en el concepto de silogismo, término que utiliza en el texto original para referirse al concepto de deducción o inferencia\cite{ross1964aristotle}. Afirma que la deducción debe ser estudiada antes que la demostración pues la primera es más general que la segunda\cite{smith1989prior}.

Esta teoría general de la deducción de Aristóteles se basa en dos ideas clave. La primera es que en ciertos casos se puede ver que una conclusión sigue lógicamente de las premisas sin que sea necesario recurrir a otras proposiciones; este tipo de deducciones reciben el nombre de silogismos \textit{inmediatos} o \textit{completos} (\textit{teleios syllogismos}), en el sentido de que no hay ninguna proposición que \textit{medie} entre las premisas y la conclusión. Estas son las que podemos considerar nuestras "reglas de inferencia". La segunda idea es que las deducciones que requieren de mediación son en realidad el resultado de encadenar silogismos inmediatos\cite{corcoran2009aristotle}.

Note to self: igual debería decidir si voy a utilizar en general "silogismo" o "deducción"

Para ilustrar su teoría general de la deducción, Aristóteles presentó un caso específico conocido como "silogismo categórico"\cite{corcoran2009aristotle}, un tipo concreto de razonamiento deductivo en el que se llega a una conclusión a partir de dos premisas. Cada uno de estos tres elementos es una oración "categórica" con tres términos, uno de los cuales (al que llama elemento \textit{central}), ocurre en cada premisa pero no en la conclusión. Aristóteles afirma que, de hecho, todos los silogismos se pueden reducir de alguna forma a silogismos categóricos\cite{smith1989prior}.

Podríamos describir el modelo de la teoría de la deducción aristotélica de la siguiente forma:

\begin{displayquote}
Todo A es un B (AaB)\\
Ningún A es un B (AeB)\\
Algún A es un B (AiB)\\
Algún A no es un B (AoB)\\
\end{displayquote}

Entonces se siguen las siguientes reglas de inferencia

Conversiones

\begin{enumerate}
    \item BeA $\implies$ AeB
    \item BiA $\implies$ AiB
    \item BaA $\implies$ AiB
\end{enumerate}

Eliminaciones universales

\begin{enumerate}
    \item AaB, BaC $\vdash$ AaC
    \item AeB, BaC $\vdash$ AeC
\end{enumerate}

Eliminaciones existenciales

\begin{enumerate}
    \item AaB, BiC $\vdash$ AiC
    \item AeB, BiC $\vdash$ AoC
\end{enumerate}

Entonces una deducción se define de manera formal de la siguiente forma: ?? no se me da pereza

Ejemplo:

\cite{corcoran2009aristotle,smith1989prior}

[falta quizás mencionar aqui algo de las Categorias de aristoteles, pero ya lo vere luego]

[falta también mencionar que Aristóteles da en los analíticos el método de la demostración por reducción al absurdo]

[falta también mencionar que el texto está inlcuido en el Organon y un resumen breve de de que se habla en cada uno de los libros, diciendo que me voy a centrar en los analíticos (y quizá en las categorías)]

\subsection{De Aristóteles hasta Boole}

En un inicio, el Organon se utilizaba como material didáctico en la escuela Peripatética, que fundó Aristóteles en el Liceo en el año 335 a. C.

En el momento de la muerte de Aristóteles, se está desarrollando también en Grecia el movimiento estoico, surgiendo la lógica estoica. Aunque los filósofos que impulsaron este sistema, principalmente Chrysippus, habían estudiado ampliamente la lógica aristotélica, y a pesar de que estas dos se consideran en conjunto los dos grandes sistemas lógcos en el mundo clásico, también menudo se consideran como disciplinas rivales.

En particular, para el estoico la unidad más pequeña no es el "silogismo" sino la "proposición" o "aserción". Las aserciones en este sistema tienen un valor de verdad asociado que depende de las condiciones en las que se expresen (por ejemplo, la aserción "es de día" solo podrá ser cierta cuando realmente sea de día). Recordemos que en el sistema aristotélico, la veracidad "real" de un silogismo no es relevante; solo nos interesa saber si podemos inferir naturalmente unos términos a partir de otros.

Además, los elementos "unidad" del sistema, es decir, las aserciones simples como "es de día", se pueden conectar con otras mediante conectores lógicos para formar aseciones complejas. Algunos de estos conectores son la implicación ("si es de día, hay luz"), la conjunción ("es de día y hay luz") y las disjunción ("o es de día o hay es de noche")\cite{algra1999cambridge}. Este sistema nos empieza a recordar al sistema de Lógica Proposicional que hoy conocemos.

A lo largo de los siguientes siglos, se realizó una gran cantidad de copias, ediciones, traducciones, reorganizaciones y comentarios sobre esta obra, lo que contribuyó a su gran influencia sobre la tradición filosófica posterior.

Una primera edición y reorganización del Organon a destacar es obra de Andrónico de Rodas, director de la escuela peripatética entre los años 78 y 47 a. C. El orden que Andrónico decidió cuidadosamente se corresponde con el que conocemos hoy\cite{hatzimichali2013texts}.

La introducción de los textos de Aristóteles a la cultura occidental se debe en gran parte a las traducciones de Boecio. Aunque al inicio del siglo VI los académicos romanos de la época aún no contaban con muchos textos científicos sobre matemáticas o medicina, ya tenían traducciones y explicaciones detalladas de la primera mitad del Organon\cite{charles2004latin}. Sin embargo, en la primera mitad de la Edad Media, la lógica no era un estudio central para los romanos y no se hicieron muchas innovaciones en esta ciencia\cite{marebon2008logic}. Tras la caída del Imperio Romano, muchos de estos trabajos se perdieron y dejaron de estar disponibles en Occidente; tendremos que esperar al siglo XII para volver a encontrar este contenido.

\subsubsection{Árabes}

Por otro lado, alrededor del año 900 d. C., el Organon ya se había traducido al árabe y sujeto a un estudio intensivo. En particular, tenemos varios textos de la escuela de Baghdad. El lógico más famoso de esta escuela fue Alfarabi (950 d.c.), que escribió una serie de tratados introductorios a la lógica además de comentarios sobre los libros del Organon.

En la obra de Alfarabi encontramos una objeto de estudio recurrente en muchos textos lógicos árabes: el intento de demostrar que todos los argumentos válidos se pueden escribir en forma de silogísmo, un intento que ya conocíamos en Aristóteles.

Después de la muerte de Alfarabi, nació otra tradición lógica distinta, con gran influencia de los textos de Avicenna (1037 dc). Aunque Avicenna se refería a Alfarabi como su predecesor, después solo de Aristóteles, su sistema de silogismos se diferenciaba en algunos puntos estructurales.

Sin embargo, medio siglo después de la muerte de Avicenna, hay una clara respuesta por parte de Al-Ghazali (1111 d.c.). En este momento de la historia de la lógica, incluso los sabios que rechazaban a filosofía griega estaban de acuerdo en que la lógica era necesaria? innegable? como sistema formal.

Las escuelas científicas, particularmente en España, siguieron enseñando la tradición de la Escuela de Baghdad, culminando con el trabajo de Averroes (1198), considerado el mayor logicista de la cultura musulmana en España (Muslim Spain). El proyecto de Averroes consistía en dar un sentido consistente a los textos aristotélicos, como extensión de los métodos de la escuela.

A pesar de que Averroes veia algunos problemas en la tradición aristotélica, los cuales describe en su obra The Essays, en todo momento de su carrera intentó preservar la perspectiva aristotélica\cite{street2001arabic}. En The Essays, escribe:

\begin{displayquote}
     \textit{Esto me ha llevado ahora (dada mi alta opinión de Aristóteles y mi creencia de que su teorización es mejor que las de todas las demás personas) a examinar esta cuestión con seridad y con gran esfuerzo.}\footnote{Mi traducción del texto ya traducido al inglés en \cite{street2001arabic}: \textit{This has led me now
    (given my high opinion of Aristotle, and my belief that his theorization is better than that of all other people) to scrutinize this question seriously and
    with great effort.}}
\end{displayquote}

A pesar de que en el mundo islámico las obras de Avicena y Algazel habían, en muchos casos, reemplazado al propio Aristóteles como referencia lógica, en Córdoba surgió un movimiento aislado, encabezado por Averroes, que buscaba regresar al pensamiento aristotélico puro. Este esfuerzo, aunque atípico en la filosofía islámica, tuvo un impacto profundo en la filosofía europea. Las obras de Averroes, como sus Grandes Comentarios, empezaron a circular entre los académicos latinos (de lengua latina) poco después de su muerte en 1198, marcando un punto de inflexión en la recepción de la lógica árabe en Occidente\cite{street2001arabic, charles2004latin}. [esto no es añadido sino posible alternativa de redacción]

\subsubsection{Occidente}

Este intento de regresar a la lógica de Aristóteles (Averroes) tuvo un impacto inmenso en la filosofía Occidental. Las primeras traducciones de las obras de Aristóteles (como los Analíticos) al Latín en esta segunda mitad de la Edad Media se deben a Jaime de Venecia (segundo cuarto del siglo XI - 1147) y son directamente del griego.

Sin embargo, la mayoría de los traductores de esta época estaban interesados únicamente en los textos ya traducidos al árabe, pues la cultura arábica era altamente dominante y avanzada en esta área. Para estos traductores, el objetivo era replicar tan fielmente como fuera posible, la filosofía arábica en Latín.

Entre ellos, destacamos a Gerardo de Cremona (1114 - 1187), que en el mismo siglo que Jaime tradujo varios comentarios árabes del Organon al Latín. Alrededor del 1220, William de Luna como tradujo los comentarios de Averroes del Isagoge, Categories, De Interpretatione y los Analíticos. Al mismo tiempo, Jacob Anatoli tradujo estos textos al hebreo\cite{charles2004latin}.

Existen algunas aportaciones nuevas a la lógica en este momento, como por ejemplo los comentarios sobre lógica aristotélica de Pedro Abelardo (1079 - 1142). Aún así, el eje central de estas aportaciones sigue siendo el Organon y el trabajo de Boecio, y consisten fundamentalemente en pequeñas ampliaciones de la lógica de silogismos. No se trata de un periodo innovador, pero sí se consolida la Lógica como disciplina fundamental de la filosofía\cite{marebon2008logic}, formando parte de las "siete artes liberales", siendo parte del \textit{Trivium} junto con la Gramática y la Retórica\cite{dutilh2008logic}.

\subsubsection{Ockham}

Quizás el primer logicista que empieza a separarse notablemente de la lógica silogista es Guillermo de Ockham (1285 - 1349). Es conocido principalmente por ser el pionero del nominalismo. En el nominalismo, se rechaza el concepto de universalidad: se considera que dos objetos denominados con el mismo término no tienen nada más en comñun a parte de esta denominación (por ejemplo, lo único que todas las sillas tienen en común es que se llaman "silla"). Esta idea, que ya había sugerido antes Boecio\cite{blackburn2005oxford} y también Abelardo, era opuesta al realismo adoptado por Platón y en parte por Aristóteles.

Otra contribución interesante de Ockham es la conocida como "navaja de Ockham" o "principio de economía"; el principio de que "las entidades no deben ser multiplicadas sin necesidad"\cite{blackburn2005oxford}. De aquí se deduce también la idea de que la explicación más simple suele ser la más acertada. Curiosamente, aunque este principio se atribuye a Ockham, es probable que tuviera sus orígenes incluso antes que Aristóteles\cite{thorburn1918myth}.

Existía por tanto un interés en formalizar la lógica de una manera menos complicada quizás que Aristóteles. Ockham recupera los conectores lógicos conjunción, disjunción y negación y da una serie de reglas de inferencia\cite{boehner1990philosophical}. Nos estamos acercando por tanto a la lógica de predicados y alejando de la de silogismos.

Además, sobre conectores lógicos, Ockham escribió lo siguiente: "También debe saberse que el opuesto contradictorio de una proposición copulativa es una proposición disyuntiva compuesta por los contradictorios de las partes de la copulativa"\cite{logicmuseum_ockham}. En lenguaje moderno, esto es $\lnot (A \land B) = \lnot A \lor \lnot B$, es decir, tenemos aquí el precursor de las Leyes de De Morgan.

\subsubsection{Siglo XVII}

Después de Ockham hubo una buena cantidad de innovaciones en lógica, consolidación de las ideas anteriores, con un interés en las falacias y las paradojas, desarrollo de las reglas de inferencia etc.

Puesto que empieza a haber una gran cantidad de innovaciones en muchas direcciones, vamos a centrarnos en aquellas que son especialmente interesantes para este trabajo.

Como hemos comentado antes, empieza a existir un interés por la formalización de la lógica. Esto es de especial interés para este trabajo puesto que, para que sea posible la computación de un objeto, es necesaria su previa formalización. Podemos mencionar a Descartes como un primer paso en esta dirección; aunque no trabajó explícitamente con esta idea, su objetivo en obras como el \textit{Discurso del Método} era clarificar ideas simples y desarollar un método deductivo concreto que pudiera revelar nuevas verdades\cite{wahl2008port}.

LEIBNIZ

La figura más importante que vamos a destacar es Leibniz.

El trabajo de Leibniz en lógica, aunque está realcionado con la teoría de silogismos, tiene como objetivo la construcción de un "cálculo universal" mucho más general. Este cálculo serviría como una herramienta para determinar que inferencias formales son lógicamente validas. Además, Leibniz quería que este cálculo se pudiera aplicar a cualquier proposición para "calcular" su veracidad de una forma puramente mecánica.

Sabemos que Leibniz elaboró varios borradores buscando esta herramienta con intención de publicarlos, pero nunca llegó a estar satisfecho con el resultado y todos sus trabajos sobre lógica se publicaron póstumamente.

El cálculo más importante desarollado por Leibniz es el "álgebra de conceptos". El punto de comienzo de este modelo es la teoría tradicional aristotélica del silogismo. Tomamos las relaciones entre dos conceptos $A$ y $B$ definidas por Aristóteles:

\begin{displayquote}
    Todo $A$ es un $B$ ($AaB$)\\
    Ningún $A$ es un $B$ ($AeB$)\\
    Algún $A$ es un $B$ ($AiB$)\\
    Algún $A$ no es un $B$ ($AoB$)\\
\end{displayquote}

En este contexto, vamos a considerar el concepto "no ser $A$" y lo denotamos como $\bar{A}$. Entonces, las expresiones anteriores se pueden escribir solo en términos de "Todo $A$ es un $B$":

\begin{displayquote}
    Todo $A$ es un $B$ ($AaB$)\\
    Todo $A$ es un $\bar{B}$ ($AeB$)\\
    $\lnot$ (Todo $A$ es un $\bar{B}$) ($AiB$)\\
    $\lnot$ (Todo $A$ es un $B$) ($AoB$)\\
\end{displayquote}

Además, Leibniz hace los siguientes cambios con respecto a la lógica silogista:


\begin{enumerate}
    \item Obvia la palabra "Todo" y escribe simplemente "$A$ es $B$" en su lugar.
    \item Considera la "conjunción conceptual" que combina dos conceptos $A$ y $B$ en yuxtaposción $AB$.
    \item Elimina las restriciones tradicionales relacionadas con el numero de premisas y el número de conceptos en las premisas de un silogismo. Es decir, se considera cualquier tipo de inferencia entre sentencias de la forma "$A$ es $B$", donde $A$ y $B$ pueden ser arbitrariamente complejos.
\end{enumerate}

El lenguaje resultante se conoce como "álgebra de conceptos" o $L1$. Una posible axiomatización de $L1$ podría hacerse utilizando solo la negación, la conjunción y la relación de contenido ("$A$ es $B$") como operadores conceptuales primitivos.

Leibniz hace un estudio extensivo de este sistema, y da una amplia cantidad de propiedades y reglas de inferencia, como por ejemplo la doble negación, la reflexividad y transitividad del contenido, etc.

Por otro lado, la lógica de Leibniz alcanza su máxima extensión con su teroía de "coneptos indefinidos", que, aunque incompleta y defectuosa, era suficientemente clara como para recontruirse sin ambiguedades. El sistema resultante tiene algunas diferencias con la lógica de segundo orden que hoy ocnocemos, pero igualmente constituye un avance hacia la formalización de la teoría de cuantificadores.

De la misma forma que el propio Leibniz no estuvo satisfecho con su trabajo, muchos logicistas posteriores criticaron su trabajo y lo consideraron inferior al de Boole. Aunque Lewis Carroll consideró que, aunque su trabajo resultó un primer boceto para la lógica simbólica, era algo positivo que Boole no conociera estas innovaciones, pues las suyas supusieron un mejor resultado. Por otro lado, H. Sauer decidió que el cálculo de Leibniz era imperfecto por "no haberse deshecho del viejo error de que todos los conceptos se pueden construir mediante conjunción de otros conceptos más simples, y de que todas las proposiciones se pueden poner de la forma 'A es B', bajo la influencia aristotélica".

No fue hasta la mitad de los años ochenta que se demostró mediante pruebas estricas que el álgebra de conceptos de Leibniz es equivalente (o isomórgfica) al álgebra de conjuntos de Boole, y que su teoría de "conceptos indefinidos" constituye una impotante anticipación a la teoría moderna de cuantificadores\cite{lenzen2004leibniz}.

Por último, brevemente destacamos a Kant, que sentó las bases de tres puntos estructurales importantes de la lógica moderna: la distinción entre concepto y objeto, la primaciía de la proposición como unidad del análisis lógico y la concepción de la lógica como estudio de la estructura de sistemas lógicos, en lugar de solo la validez de inferencias específicas\cite{tiles2004kant}.


\section{Lógica simbólica}

La matematización de la lógica? Boole y Frege


\section{Siglos XIX, XX}

\subsection{Boole}

Orden cronológico de las cosas que están pasando aquí:

1. Lógica simbólica: desarrollo de la
lógica simbólica en los siglos XIX y XX, momento en el que ocurren varias
innovaciones, como por ejemplo la renovación de la lógica formal (Boole, De
Morgan), la aparición de la teoría de conjuntos (Cantor), la teoría de tipos
(Russel)

2. 1930: Kurt Gödel proved the Completeness Theorem, which states that if a formula is logically valid in first-order logic, it is provable in its formal system. This was part of his doctoral dissertation.

3. 1936: Alan Turing introduced the concept of the Turing machine in his paper "On Computable Numbers, with an Application to the Entscheidungsproblem (= Decidability problem)", laying the foundation for computability theory.

4. Apparition of Logic Programming (Planner and Prolog) 
1967: Planner, developed by Carl Hewitt, introduced concepts of logical inference and goal-directed reasoning., Prolog in 1972

5. 1971: Stephen Cook introduced the concept of NP in his landmark paper, "The Complexity of Theorem-Proving Procedures". This formalized the class of decision problems verifiable in polynomial time by a nondeterministic Turing machine.

Yo creo que este es un buen timeline y sería ir desarrollando estos conceptos y tal de forma que se vayan conectando entre sí y al concepto de la lógica



Sería interesante hablar de :


\section{Theorem Provers??}
- Tecton
- Lean


% Bibliografía

\renewcommand{\refname}{Referencias}
\bibliographystyle{plain}
\bibliography{references.bib}

\end{document}
